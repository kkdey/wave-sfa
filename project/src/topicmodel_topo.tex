\documentclass[]{article}
\usepackage{lmodern}
\usepackage{amssymb,amsmath}
\usepackage{ifxetex,ifluatex}
\usepackage{fixltx2e} % provides \textsubscript
\ifnum 0\ifxetex 1\fi\ifluatex 1\fi=0 % if pdftex
  \usepackage[T1]{fontenc}
  \usepackage[utf8]{inputenc}
\else % if luatex or xelatex
  \ifxetex
    \usepackage{mathspec}
    \usepackage{xltxtra,xunicode}
  \else
    \usepackage{fontspec}
  \fi
  \defaultfontfeatures{Mapping=tex-text,Scale=MatchLowercase}
  \newcommand{\euro}{€}
\fi
% use upquote if available, for straight quotes in verbatim environments
\IfFileExists{upquote.sty}{\usepackage{upquote}}{}
% use microtype if available
\IfFileExists{microtype.sty}{%
\usepackage{microtype}
\UseMicrotypeSet[protrusion]{basicmath} % disable protrusion for tt fonts
}{}
\usepackage[margin=1in]{geometry}
\usepackage{graphicx}
\makeatletter
\def\maxwidth{\ifdim\Gin@nat@width>\linewidth\linewidth\else\Gin@nat@width\fi}
\def\maxheight{\ifdim\Gin@nat@height>\textheight\textheight\else\Gin@nat@height\fi}
\makeatother
% Scale images if necessary, so that they will not overflow the page
% margins by default, and it is still possible to overwrite the defaults
% using explicit options in \includegraphics[width, height, ...]{}
\setkeys{Gin}{width=\maxwidth,height=\maxheight,keepaspectratio}
\ifxetex
  \usepackage[setpagesize=false, % page size defined by xetex
              unicode=false, % unicode breaks when used with xetex
              xetex]{hyperref}
\else
  \usepackage[unicode=true]{hyperref}
\fi
\hypersetup{breaklinks=true,
            bookmarks=true,
            pdfauthor={Kushal K Dey},
            pdftitle={Topographical Topic Models},
            colorlinks=true,
            citecolor=blue,
            urlcolor=blue,
            linkcolor=magenta,
            pdfborder={0 0 0}}
\urlstyle{same}  % don't use monospace font for urls
\setlength{\parindent}{0pt}
\setlength{\parskip}{6pt plus 2pt minus 1pt}
\setlength{\emergencystretch}{3em}  % prevent overfull lines
\setcounter{secnumdepth}{0}

%%% Use protect on footnotes to avoid problems with footnotes in titles
\let\rmarkdownfootnote\footnote%
\def\footnote{\protect\rmarkdownfootnote}

%%% Change title format to be more compact
\usepackage{titling}

% Create subtitle command for use in maketitle
\newcommand{\subtitle}[1]{
  \posttitle{
    \begin{center}\large#1\end{center}
    }
}

\setlength{\droptitle}{-2em}
  \title{Topographical Topic Models}
  \pretitle{\vspace{\droptitle}\centering\huge}
  \posttitle{\par}
  \author{Kushal K Dey}
  \preauthor{\centering\large\emph}
  \postauthor{\par}
  \predate{\centering\large\emph}
  \postdate{\par}
  \date{January 24, 2016}



\begin{document}

\maketitle


\subsection{Introduction}\label{introduction}

In phylogenetic studies, one often encounters assemblage maps that plot
counts/abundances of different bird species on geographical maps.
Recently, bird species abundance data was collected from 35 forest spots
in Eastern and Western Himalayas by Trevor Price's lab. Assemblage map
was then computed for the bird species in each forest spot depending on
the presence/absence and/or the relative abundance patterns. The idea
then is to cluster these assemblage maps using a graded membership model
and represent each map by a weighted combination of a set of base maps.

\subsection{Model}\label{model}

Let us assume that each map is a grid with \(R\) rows and \(C\) columns.
Therefore there are \(V = R \times C\) cells/ voxels. We can pool the
\(V\) cells/voxels in the assemblage map and view them as as features.
Each cell/voxel contains a count value representing the total number of
species present in that cell or the total abundance of all species in
that cell. Let \(c_{nv}\) be the counts of reads number of bird species/
total number of birds observed across different species in forest spot
\(n\) and cell/voxel \(v\). Let \(c_{n+}\) be the sum of the counts
across the cells for forest spot \(n\) (this would be equal to the total
counts in the \(n\)th map corresponding to the assemblage map for \(n\)
th forest spot).

\[ (c_{n1}, c_{n2}, \cdots, c_{nV}) \sim Mult (c_{n+}, p_{n1}, p_{n2}, \cdots, p_{nV})  \]

\[ p_{nv} = \sum_{k=1}^{K} \omega_{nk} f_{kv} \hspace{1 cm} \sum_{k} \omega_{nk} =1 \hspace{0.5 cm} \forall n \hspace{1 cm} \sum_{v} f_{kv} =1 \hspace{0.5 cm} \forall k\]

Here \(\omega_{n.}\) represents the topic proportions for \(n\) th
forest spot. On the other hand \(f_{k.}\) represents the probability
distribution or relative intensity across the different cells for the
\(k\) th cluster or topic. Note that \(f_{k.}\) may also be visualized
as the base image for the \(k\) th cluster/topic. We assume the priors

\[ (\omega_{n1}, \omega_{n2}, \cdots, \omega_{nK}) \sim  Dir_{K} \left ( \frac{1}{K}, \frac{1}{K}, \cdots, \frac{1}{K} \right) \hspace{1 cm} \forall n\]

\[ (f_{k1}, f_{k2}, \cdots, f_{kV}) \sim Dir_{V} \left ( \alpha_{k1}, \alpha_{k2}, \cdots, \alpha_{kV} \right ) \]

We define

\begin{equation}
 \alpha_{kv} : = \frac{exp(- \frac{|| r_{v} - \mu_{k} ||^2}{\lambda_{k}})}{\sum_{v} exp(- \frac{|| r_{v} - \mu_{k} ||^2}{\lambda_{k}})} 
 \label{lab:alpha}
 \end{equation}

This function is called \emph{Gaussian radial basis function}. Note that
by definition \(\sum_{v} \alpha_{v}=1\) and this can be viewed as the
weight we are assigning to each cell or voxel for the \(k\) th base
map/cluster.

We assume \(\mu_{k}\) and \(\lambda_{k}\) to be hyperparameters in the
model although in TFA, prior distributions are assumed for these two
parameters.

\subsection{Model Specifications}\label{model-specifications}

We can assume that

\begin{equation}
c_{n+} \sim Poi(\lambda_{n}) 
\label{lab:libsize}
\end{equation}

Then given the model specifications, we get

\begin{equation}
c_{nv} \sim Poi \left ( \lambda_{n} \sum_{k} \omega_{nk} f_{kv} \right)
\end{equation}

Let \(z_{nkv}\) represents the number of counts from forest spot \(n\)
and cell \(v\) that comes from \(k\) th subgroup or base map. By
definition,

\[ \sum_{k=1}^{K} z_{nkv} = c_{nv}  \]

Since the summation of two independent Poisson random variables is also
a Poisson variable with mean equal to the sum of the means of the
original random variables, we can infer that

\[ z_{nkv} \sim Poi \left (\lambda_{n}\omega_{nk} f_{kv} \right ) \]

We define

\[ z_{kv} := \sum_{n} z_{nkv} \]

\(z_{kv}\) is the number of counts of bird species corresponding to
\(k\) th base map and \(v\) th cell. We can write

\[ z_{kv} \sim Poi \left ( f_{kv} \sum_{n} \lambda_{n} \omega_{nk} \right) \]

\subsection{Model Estimation}\label{model-estimation}

We start at iteration \(n\) and suppose we have the iterates
\(\mu^{(m)}_{k}\), \(\lambda^{(m)}_{k}\), \(\omega^{(m)}_{nk}\) and
\(f^{(m)}_{kv}\). We want to update the parameters for the \(m+1\) th
step. We update \(f_{kv}\) using the EM algorithm.

\begin{equation}
\mathcal{Q} \left ( f_{k.} | C_{N \times G}, f^{(m)}, \mu^{(m)}_{k}, \lambda^{(m)}_{k}, \omega^{(m)} \right ) = \mathbb{E}_{Z | C_{N \times G}, f^{(m)}, \mu^{(m)}_{k}, \lambda^{(m)}_{k}, \omega^{(m)}} \left ( log \; \mathcal{L} (z_{k1}, z_{k2}, \cdots, z_{kV} | f_{k.})  + log \; \pi(f_{k.} | \mu^{(m)}_{k.}, \lambda^{(m)}_{k.}) \right )
\label{lab:estep}
\end{equation}

where

\begin{equation}
\pi(f_{k.} | \mu_{k}, \lambda_{k} ) \propto \prod_{v=1}^{V} f_{kv}^{\alpha_{kv}}
\label{lab:prior}
\end{equation}

where \(\alpha_{kv}\) is defined as per Equation \ref{lab:alpha}.

and

\begin{equation}
\mathcal{L} (z_{k1}, z_{k2}, \cdots, z_{kV} | f_{k.}) := Mult \left (z_{k+}, f_{k1}, f_{k2}, \cdots, f_{kV} \right)
\label{lab:loglik}
\end{equation}

Using EM algorithm, the parameter updates we get are

\begin{equation}
f^{(m+1)}_{kv} : = \frac{\mathbb{E} \left ( z_{kv} |  C_{N \times V}, f^{(m)}, \mu^{(m)}_{k}, \lambda^{(m)}_{k}, \omega^{(m)} \right) + \alpha^{(m)}_{kv}}{\sum_{v} \mathbb{E} \left ( z_{kv} |  C_{N \times V}, f^{(m)}, \mu^{(m)}_{k}, \lambda^{(m)}_{k}, \omega^{(m)} \right) + \sum_{v} \alpha^{(m)}_{kv}}
\label{lab:mstep}
\end{equation}

where

\[ \mathbb{E} \left ( z_{kv} |  C_{N \times V}, f^{(m)}, \mu^{(m)}_{k}, \lambda^{(m)}_{k}, \omega^{(m)} \right) : = c_{nv} \frac{\omega^{(m)}_{nk} f^{(m)}_{kv}}{\sum_{h=1}^{K} \omega^{(m)}_{nh} f^{(m)}_{hv}} \]

Given that we have the new updates \(f^{(m+1)}_{kv}\), we can obtain new
estimates for \(\mu^{(m+1)}_{k}\) and \(\lambda^{(m+1)}_{k}\). We define

\[ \mu^{(m+1)}_{k} = \frac{\sum_{v} v f^{(m)}_{kv}}{\sum_{v} f^{(m)}_{kv}} \]

\[ \lambda^{(m+1)}_{k} = \frac{\sum_{v} v^2 f^{(m)}_{kv}}{\sum_{v} f^{(m)}_{kv}} - (\mu^{(m)}_{k})^2 \]

Given the new updates for \(f\), we can update the \(\omega\) parameters
in the same way as done by Matt Taddy in his
\href{http://arxiv.org/pdf/1109.4518v3.pdf}{paper} using active set
strategy and convex optimization.

At the end of these steps, we will have \(\omega^{(m+1)}_{nk}\),
\(f^{(m+1)}_{kv}\), \(\mu^{(m+1)}_{k}\) and \(\lambda^{(m+1)}_{k}\). We
can use these to update the parameters further and we continue till the
log-likelihood converges.

\end{document}
