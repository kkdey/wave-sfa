\documentclass[]{article}
\usepackage{lmodern}
\usepackage{amssymb,amsmath}
\usepackage{ifxetex,ifluatex}
\usepackage{fixltx2e} % provides \textsubscript
\ifnum 0\ifxetex 1\fi\ifluatex 1\fi=0 % if pdftex
  \usepackage[T1]{fontenc}
  \usepackage[utf8]{inputenc}
\else % if luatex or xelatex
  \ifxetex
    \usepackage{mathspec}
    \usepackage{xltxtra,xunicode}
  \else
    \usepackage{fontspec}
  \fi
  \defaultfontfeatures{Mapping=tex-text,Scale=MatchLowercase}
  \newcommand{\euro}{€}
\fi
% use upquote if available, for straight quotes in verbatim environments
\IfFileExists{upquote.sty}{\usepackage{upquote}}{}
% use microtype if available
\IfFileExists{microtype.sty}{%
\usepackage{microtype}
\UseMicrotypeSet[protrusion]{basicmath} % disable protrusion for tt fonts
}{}
\usepackage[margin=1in]{geometry}
\usepackage{graphicx}
\makeatletter
\def\maxwidth{\ifdim\Gin@nat@width>\linewidth\linewidth\else\Gin@nat@width\fi}
\def\maxheight{\ifdim\Gin@nat@height>\textheight\textheight\else\Gin@nat@height\fi}
\makeatother
% Scale images if necessary, so that they will not overflow the page
% margins by default, and it is still possible to overwrite the defaults
% using explicit options in \includegraphics[width, height, ...]{}
\setkeys{Gin}{width=\maxwidth,height=\maxheight,keepaspectratio}
\ifxetex
  \usepackage[setpagesize=false, % page size defined by xetex
              unicode=false, % unicode breaks when used with xetex
              xetex]{hyperref}
\else
  \usepackage[unicode=true]{hyperref}
\fi
\hypersetup{breaklinks=true,
            bookmarks=true,
            pdfauthor={Kushal K Dey},
            pdftitle={Topographical Topic Models},
            colorlinks=true,
            citecolor=blue,
            urlcolor=blue,
            linkcolor=magenta,
            pdfborder={0 0 0}}
\urlstyle{same}  % don't use monospace font for urls
\setlength{\parindent}{0pt}
\setlength{\parskip}{6pt plus 2pt minus 1pt}
\setlength{\emergencystretch}{3em}  % prevent overfull lines
\setcounter{secnumdepth}{0}

%%% Use protect on footnotes to avoid problems with footnotes in titles
\let\rmarkdownfootnote\footnote%
\def\footnote{\protect\rmarkdownfootnote}

%%% Change title format to be more compact
\usepackage{titling}

% Create subtitle command for use in maketitle
\newcommand{\subtitle}[1]{
  \posttitle{
    \begin{center}\large#1\end{center}
    }
}

\setlength{\droptitle}{-2em}
  \title{Topographical Topic Models}
  \pretitle{\vspace{\droptitle}\centering\huge}
  \posttitle{\par}
  \author{Kushal K Dey}
  \preauthor{\centering\large\emph}
  \postauthor{\par}
  \predate{\centering\large\emph}
  \postdate{\par}
  \date{January 24, 2016}



\begin{document}

\maketitle


\subsection{Introduction}\label{introduction}

In phylogenetic studies, one often encounters assemblage maps that plot
counts/abundances of different bird species on geographical maps.
Recently, bird species abundance data was collected from 35 forest spots
in Eastern and Western Himalayas by Trevor Price's lab. Assemblage map
was then computed for the bird species in each forest spot depending on
the presence/absence and/or the relative abundance patterns. The idea
then is to cluster these assemblage maps using a graded membership model
and represent each map by a weighted combination of a set of base maps.

\subsection{Model}\label{model}

Let us assume that each map is a grid with \(R\) rows and \(C\) columns.
Therefore there are \(V = R \times C\) cells/ voxels. We can pool the
\(V\) pixels/voxels in the assemblage map and view them as as features
and each pixel/voxel contains a count value representing the abundances.
Let \(c_{nv}\) be the counts of reads for sample \(n\) and voxel \(v\).
Let \(c_{n+}\) be the sum of reads for sample \(n\), also called the
\emph{library size}.

\[ (c_{n1}, c_{n2}, \cdots, c_{nV}) \sim Mult (c_{n+}, p_{n1}, p_{n2}, \cdots, p_{nV})  \]

\[ p_{ng} = \sum_{k=1}^{K} \omega_{nk} \theta_{kg} \hspace{1 cm} \sum_{k} \omega_{nk} =1 \hspace{0.5 cm} \forall n \hspace{1 cm} \sum_{g} \theta_{kg} =1 \hspace{0.5 cm} \forall k\]

Here \(\omega_{n.}\) represents the topic proportions for \(n\) th
samples. On the other hand \(\theta_{k.}\) represents the probability
distribution on the genes for the \(k\)th topic or cluster.

\end{document}
