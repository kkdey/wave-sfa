\documentclass[]{article}
\usepackage{lmodern}
\usepackage{amssymb,amsmath}
\usepackage{ifxetex,ifluatex}
\usepackage{fixltx2e} % provides \textsubscript
\ifnum 0\ifxetex 1\fi\ifluatex 1\fi=0 % if pdftex
  \usepackage[T1]{fontenc}
  \usepackage[utf8]{inputenc}
\else % if luatex or xelatex
  \ifxetex
    \usepackage{mathspec}
    \usepackage{xltxtra,xunicode}
  \else
    \usepackage{fontspec}
  \fi
  \defaultfontfeatures{Mapping=tex-text,Scale=MatchLowercase}
  \newcommand{\euro}{€}
\fi
% use upquote if available, for straight quotes in verbatim environments
\IfFileExists{upquote.sty}{\usepackage{upquote}}{}
% use microtype if available
\IfFileExists{microtype.sty}{%
\usepackage{microtype}
\UseMicrotypeSet[protrusion]{basicmath} % disable protrusion for tt fonts
}{}
\usepackage[margin=1in]{geometry}
\usepackage{graphicx}
\makeatletter
\def\maxwidth{\ifdim\Gin@nat@width>\linewidth\linewidth\else\Gin@nat@width\fi}
\def\maxheight{\ifdim\Gin@nat@height>\textheight\textheight\else\Gin@nat@height\fi}
\makeatother
% Scale images if necessary, so that they will not overflow the page
% margins by default, and it is still possible to overwrite the defaults
% using explicit options in \includegraphics[width, height, ...]{}
\setkeys{Gin}{width=\maxwidth,height=\maxheight,keepaspectratio}
\ifxetex
  \usepackage[setpagesize=false, % page size defined by xetex
              unicode=false, % unicode breaks when used with xetex
              xetex]{hyperref}
\else
  \usepackage[unicode=true]{hyperref}
\fi
\hypersetup{breaklinks=true,
            bookmarks=true,
            pdfauthor={Kushal K Dey},
            pdftitle={Topic model with Batch effects},
            colorlinks=true,
            citecolor=blue,
            urlcolor=blue,
            linkcolor=magenta,
            pdfborder={0 0 0}}
\urlstyle{same}  % don't use monospace font for urls
\setlength{\parindent}{0pt}
\setlength{\parskip}{6pt plus 2pt minus 1pt}
\setlength{\emergencystretch}{3em}  % prevent overfull lines
\setcounter{secnumdepth}{0}

%%% Use protect on footnotes to avoid problems with footnotes in titles
\let\rmarkdownfootnote\footnote%
\def\footnote{\protect\rmarkdownfootnote}

%%% Change title format to be more compact
\usepackage{titling}

% Create subtitle command for use in maketitle
\newcommand{\subtitle}[1]{
  \posttitle{
    \begin{center}\large#1\end{center}
    }
}

\setlength{\droptitle}{-2em}
  \title{Topic model with Batch effects}
  \pretitle{\vspace{\droptitle}\centering\huge}
  \posttitle{\par}
  \author{Kushal K Dey}
  \preauthor{\centering\large\emph}
  \postauthor{\par}
  \predate{\centering\large\emph}
  \postdate{\par}
  \date{January 22, 2016}



\begin{document}

\maketitle


\subsection{Introduction}\label{introduction}

In RNA-seq experiments, we often encounter samples coming from different
batches. The batches may be determined by the amplification procedures
used, or the sequencing machine or even the sequencing lane effects.
When these batch effects or technical effects are present in the
samples, it becomes difficult to often separate out the biological
information from the technical information (the latter is often
relatively stronger). The topic model or the grade-of membership model
has been used to cluster the samples based on their RNA-seq reads counts
data (see
\href{https://github.com/stephenslab/count-clustering/blob/master/docs/main.pdf}{paper}).
In the paper, we have shown that the topic model is sensitive to the
presence of batch effects, however we have not been able to present a
solution to that problem. We address the issue of how one can tackle
batch effects in a topic model type framework.

We first present the standard topic model framework

\subsection{Standard Topic Model}\label{standard-topic-model}

Let \(c_{ng}\) be the counts of reads for sample \(n\) and gene \(g\).
Let \(c_{n+}\) be the sum of reads for sample \(n\), also called the
\emph{library size}.

\[ (c_{n1}, c_{n2}, \cdots, c_{nG}) \sim Mult (c_{n+}, p_{n1}, p_{n2}, \cdots, p_{nG})  \]

\[ p_{ng} = \sum_{k=1}^{K} \omega_{nk} \theta_{kg} \hspace{1 cm} \sum_{k} \omega_{nk} =1 \hspace{0.5 cm} \forall n \hspace{1 cm} \sum_{g} \theta_{kg} =1 \hspace{0.5 cm} \forall k\]

Here \(\omega_{n.}\) represents the topic proportions for \(n\) th
samples. On the other hand \(\theta_{k.}\) represents the probability
distribution on the genes for the \(k\)th topic or cluster.

\subsection{Topic model with Batch
effects}\label{topic-model-with-batch-effects}

One way batch effects may be incorporated in the above model would be to
make the topic distribution for each cluster/ topic a function of the
batch the sample is coming from, as well. Then we can write the above
model as

\begin{equation}
(c_{n1}, c_{n2}, \cdots, c_{nG}) \sim Mult (c_{n+}, p_{n1}, p_{n2}, \cdots, p_{nG})
\label{lab:mult}
\end{equation}

\begin{equation}
p_{ng} = \sum_{k=1}^{K} \omega_{nk} \theta_{b(n):k,g} \hspace{1 cm} \sum_{k} \omega_{nk} =1 \hspace{0.5 cm} \forall n \hspace{1 cm} \sum_{g} \theta_{b(n):k,g} =1 \hspace{0.5 cm} \forall k, \hspace{0.5 cm} b(n) \in \{1,2,\cdots, B \}
\end{equation}

\subsection{Prior Specification}\label{prior-specification}

Note that the above the model is analogous to applying topic model
separately for each batch. The problem with that approach is that we
will not be able to track which cluster is Batch 1 corresponds to that
cluster in Batch 2. Also, we expect each cluster distribution to have
some common features across different batches despite getting effected
by batch effects. In order to tackle this, we make the following
assumption.

For each cluster \(k\)

\begin{equation}
(\theta_{b:k, 1}, \theta_{b:k, 2}, \cdots, \theta_{b:k, G}) \sim Dir_{G} \left ( \theta_{k1}, \theta_{k2}, \cdots, \theta_{kG} \right ) \hspace{1 cm} b \in \{1,2, \cdots, B \} 
\label{lab:prior}
\end{equation}

Which is same as saying that for each batch, we are generating a sample
from the cluster with mean
\((\theta_{k1}, \theta_{k2}, \cdots, \theta_{kG})\), which represents
the cluster \(k\). This is analogous to the assumption in the normal
linear models with batch effects,

\[ y_{ng} = \mu_{t(n):b(n),g} = \mu + \tau_{t(n)} + \beta_{b(n)} + e_{ng} \]

where \(t(n)\) in the treatment effect and \(b(n)\) is the batch effect.
We often assume that

\[ \beta_{b} \sim N(0, \sigma^{2}_{b}) \]

Then

\[ \mu_{t(n):b(n)} \sim N (\mu + \tau_{n}, \sigma^{2}_{b})  : = N(\mu_{t(n)}, \sigma^{2}_{b}) \]

You can see that the treatment effects under the different batches are
then a random sample from a distribution whose mean is the treatment
effect without the batch information. Note that \(\sigma_b\) term is
there in normal models to tune the variance for each effect. We can also
put such a scaling parameter in our model Equation \ref{lab:prior}.

Then for each \(k\),

\[ (\theta_{b:k, 1}, \theta_{b:k, 2}, \cdots, \theta_{b:k, G}) \sim Dir_{G} \left ( \alpha_{b} \theta_{k1}, \alpha_{b} \theta_{k2}, \cdots, \alpha_{b} \theta_{kG} \right ) \hspace{1 cm} b \in \{1,2, \cdots, B \} \]

However, as of now, I am assuming that \(\alpha_b =1\) for all batches
and working with the simpler model.

We assume a prior for \(\theta_{kg}\).

\begin{equation}
(\theta_{k1}, \theta_{k2}, \cdots, \theta_{kG}) \sim Dir_{G} \left ( \frac{1}{KG}, \frac{1}{KG}, \cdots, \frac{1}{KG} \right) \hspace{1 cm} \forall k
\label{lab:prior2}
\end{equation}

So, essentially we have a hierarchical structure in the \(\theta\)'s, on
combining Equation \ref{lab:prior} and Equation \ref{lab:prior2}.

\includegraphics{../figs/hierarchy_batch.jpeg} We can assume the same
prior for \(\omega\) as in standard topic model, given by

\[ (\omega_{n1}, \omega_{n2}, \cdots, \omega_{nK}) \sim  Dir_{K} \left ( \frac{1}{K}, \frac{1}{K}, \cdots, \frac{1}{K} \right) \hspace{1 cm} \forall n\]

\subsection{Model estimation}\label{model-estimation}

We can assume that

\begin{equation}
c_{n+} \sim Poi(\lambda_{n}) 
\label{lab:libsize}
\end{equation}

Then combining Equation \ref{lab:mult} and Equation \ref{lab:libsize},
we get

\begin{equation}
c_{ng} \sim Poi \left ( \lambda_{n} \sum_{k} \omega_{nk} \theta_{kg} \right)
\end{equation}

Let \(z_{nkg}\) represents the number of counts from sample \(n\) and
from feature \(g\) that comes from \(k\) th subgroup or cluster. By
definition,

\[ \sum_{k=1}^{K} z_{nkg} = c_{ng}  \]

Since the summation of two independent Poisson random variables is also
a Poisson variable with mean equal to the sum of the means of the
original random variables, we can infer that

\[ z_{nkg} \sim Poi \left (\lambda_{n}\omega_{nk} \theta_{b(n):k,g} \right ) \]

Let \(z_{bkg}\) be the latent variable representing the number of reads
coming from the \(b\) th batch, \(k\) th subgroup/ cluster and gene
\(g\).

\[ z_{bkg} | \theta_{b:k,g} \sim Poi \left (\theta_{b:k,g} \sum_{b(n):b} \omega_{nk}\lambda_{n} \right )  \]

\[ z_{kg} | \theta_{b:k,g} \sim Poi \left (\sum_{b} \theta_{b:k,g} \sum_{b(n):b} \omega_{nk}\lambda_{n} \right ) \]

We can write

\begin{align*}
E(z_{kg} | \theta_{kg}) & = E \left ( \sum_{b} E \left ( z_{bkg} | \theta_{b:k,g} \right)| \theta_{kg} \right ) \\
                        & = E \left ( \sum_{b} \theta_{b:k,g} \sum_{b(n):b} \omega_{nk} \lambda_{n} |  \theta_{kg} \right) \\
                        & = \theta_{kg} \sum_{n} \omega_{nk} \lambda_{n} 
\end{align*}

Note that under the above set of assumptions, the distribution

\begin{equation}
z_{bkg} | \theta_{kg} \sim \prod_{b=1}^{B} \frac{\Gamma (\sum_{g} \theta_{kg})}{\Gamma (\sum_{g} z_{bkg} + \sum_{g} \theta_{kg})} \prod_{g=1}^{G} \frac{\Gamma (z_{bkg} + \theta_{kg})}{\Gamma (\theta_{kg})} \theta^{1/KG}_{kg} 
\label{lab:distr1}
\end{equation}

and

\begin{equation}
z_{kg} = \sum_{b} z_{bkg}
\label{lab:distr2}
\end{equation}

This leads to a completed distribution for \(z_{kg}\). this density is
difficult to handle and will be time expensive to solve for \(\theta\)
using this density function. Therefore, we resort to a more simplified
approach, through EM algorithm to obtain parameter updates at each step.

\subsection{EM algorithm}\label{em-algorithm}

Suppose at the end of update \(n\), the current estimates we have are
\(\omega^{(m)}_{nk}\) and \(\theta^{(m)}_{kg}\). We use these iterates
to obtain refined estimates \(\omega^{(m+1)}_{nk}\) and
\(\theta^{(m+1)}_{kg}\). This idea is similar to the standard topic
model iterative scheme. The only major difference is that there are
intermediate parameters containing batch effect information, namely
\(\theta^{(m+1)}_{b(n):k,g}\).

\begin{equation}
\mathcal{L} (z_{bkg} | \theta_{b:k,g}) := Poi \left (\theta_{b:k,g} \sum_{b(n):b} \omega_{nk}\lambda_{n} \right )  
\label{lab:loglik}
\end{equation}

or

\begin{equation}
\mathcal{L} (z_{bk1}, z_{bk2}, \cdots, z_{bkG} | \theta_{b:k,.}) := Mult \left (z_{bk+}, \theta_{b:k,1}, \theta_{b:k,2}, \cdots, \theta_{b:k,G} \right)
\label{lab:loglik}
\end{equation}

\begin{equation}
\pi(\theta_{b:k,.} | \theta_{k.} ) \propto \prod_{g=1}^{G} \theta_{b:k,g}^{\theta_{kg}}
\label{lab:prior}
\end{equation}

We define the E-step of the EM algorithm as follows (done separately for
each \(k\))

\begin{equation}
\mathcal{Q} \left ( \theta_{b:k,.} | C_{N \times G}, \theta^{(m)}_{b:k,.}, \theta^{(m)}_{k.} , \omega^{(m)} \right ) = \mathbb{E}_{Z | C_{N \times G}, \theta^{(m)}_{b:k,.}, \theta^{(m)}_{k.} , \omega^{(m)}} \left ( log \mathcal{L} (z_{bk1}, z_{bk2}, \cdots, z_{bkG} | \theta_{b:k,.})  + log \pi(\theta_{b:k,.} | \theta^{(m)}_{k.} ) \right )
\label{lab:estep}
\end{equation}

We next perform the M-step and we obtain the following solutions for
\(\theta_{b:k,g}\)s.

\begin{equation}
\theta^{(m+1)}_{b:k,g} : = \frac{\mathbb{E} \left ( z_{b:k,g} |  C_{N \times G}, \theta^{(m)}_{k.} , \omega^{(m)} \right) + \theta^{(m)}_{kg}}{\sum_{g} \mathbb{E} \left ( z_{b:k,g} |  C_{N \times G}, \theta^{(m)}_{k.} , \omega^{(m)} \right) + 1}
\label{lab:mstep}
\end{equation}

The expectation over
\([ z_{b:k,g} | C_{N \times G}, \theta^{(m)}_{b:k,.}, \theta^{(m)}_{k.} , \omega^{(m)} ]\)
is given by the following

\[ \mathbb{E} \left ( z_{b:k,g} |  C_{N \times G}, \theta^{(m)}_{b:k,.}, \theta^{(m)}_{k.} , \omega^{(m)} \right) : = c_{ng} \frac{\omega^{(m)}_{nk} \theta^{(m)}_{b:k,g}}{\sum_{h=1}^{K} \omega^{(m)}_{nh} \theta^{(m)}_{b:h,g}} \]

Once we obtain the estimates \(\theta^{(m+1)}_{b:k,g}\), then we would
like to update \(\theta^{(m+1)}_{kg}\) conditional on
\(\theta^{(m+1)}_{b:k,g}\) for all \(b \in \{1,2, \cdots, B \}\).

One can assume \(\theta^{(m+1)}_{b:k,.}\) for all \(b\) to be a random
sample of size \(B\) (same as number of batches) given \(\theta_{k.}\)
and given the data, we want to estimate the parameters \(\theta_{k.}\).
However estimating MLE of the Dirichlet parameters is not easy, and
requires Newton-Raphson Method (check
\href{http://www.msr-waypoint.com/en-us/um/people/minka/papers/dirichlet/minka-dirichlet.pdf}{paper}).
One can obtain a MOM (Methods of Moments) type estimator with a tuning
parameter \(\nu\) as follows.

\[ \theta^{(m+1)}_{kg} = \frac{1}{B+\nu} \sum_{b=1}^{B} \theta^{(m+1)}_{b:k,g}  + \frac{\nu}{B+\nu} \frac{1}{KG} \]

instead of the MLE estimator derived from the conditional distribution
of \(z_{kg}\) given \(\theta_{kg}\) as given in Equation
\ref{lab:distr1} and Equation \ref{lab:distr2}. However as discussed
earlier, the original conditional distribution of
\(z_{kg} | \theta_{kg}\) seems difficult to handle.

We fix the batch \(b\). Then given \(\theta^{(m+1)}_{b:k,g}\), we can
estimate \(\omega^{(m+1)}_{nk}\) from \(\omega^{(m)}_{nk}\) and
\(\theta^{(m+1)}_{b:k,g}\) using similar convex optimization technique
used by Matt Taddy in his
\href{http://arxiv.org/pdf/1109.4518v3.pdf}{paper}.

At the end of these steps, we will have \(\omega^{(m+1)}_{nk}\),
\(\theta^{(m+1)}_{b(n):k,g}\) and \(\theta^{(m+1)}_{kg}\). We can these
use these to update the parameters further and we continue till the
log-likelihood converges.

\end{document}
